
\noindent
\subsection{Motivation}
The field of astrophysics is a relatively young field of study compared to most other disciplines of science, but in many ways it is the most fundamental. From the tiniest quantum fluctuations at the beginning of time, to the galaxy clusters found in our present day Universe, astrophysicists have to cover a range of magnitudes from the smallest particles discovered to the largest structures in the Universe. 

In this project galaxies are the focus of study. Theories for how galaxies formed and evolved since the Big Bang have been proposed since they were first discovered, and as new data and new understandings of physics emerge, new theories take over for old ones. The model that has been established as the one currently best able to explain observations of the Universe is the Lambda Cold Dark matter ($\Lambda$CDM) model. In this model, the energy in the Universe is made up of about 75 percent dark energy (one theory is that this is the so-called vacuum energy that is pushing the expansion of the Universe), 21 percent dark matter and about 4 percent baryonic (visible) matter \parencite{Planck2016}. 

There are many theories for what dark matter actually is \parencite[see e.g.,][]{Boveia2018}, but what we do know is that cosmological models require the presence of dark matter to reproduce the structures seen today. Dark matter does not interact with any particles except through gravity. In the $\Lambda$CDM model of our Universe, galaxies are located in the center of dark matter halos (hereafter, halos), which extend much further than the actual visible galaxy. Many of the properties of galaxies are linked to its host halo.

Hydrodynamical cosmological simulations have been around since the 1980s, starting as N-body simulations of only dark matter particles with a set of initial conditions \parencite{Frenk1983}. As computers became more powerful, and physicists learned more about the complicated physics of galaxies, the simulations started to incorporate stars, gas and other baryonic components. The resolution and size of simulations have increased tremendously. Now it is possible to have mass resolutions that show the inner structure of galaxies and at the same time have a simulation volume that is large enough to be relevant on cosmological scales. In this respect, projects such as the Illustris and EAGLE simulations have pushed the boundaries of modern astrophysics. IllustrisTNG is the new and improved version of the Illustris simulation. The first result-papers were published in 2017, and more data is still being produced. It increases the resolution, size and amounts of physics included, to produce the largest, most detailed simulated Universe to this date. 

The use of the data from numerical simulations might seem straightforward, but comparisons against observational data or other numerical simlations require careful considerations. There are many exisitng practices for how the data is post-processed after the simulation is run, and the way that properties are defined and calculated are important factors to consider. In this thesis, the practice of using pre-calculated IllustrisTNG data from the SUBFIND group catalogues is compared against several other methods of treating the data during post-processing. Then the mock galaxy properties derived from the TNG simulation is compared against observational data, to study its efficiency in simulating real galaxy properties.

\subsection{The structure of this report}
Section \ref{theory} explains the physics of the main galaxy property relations that are covered in this report. It also contains a glossary with explanation of notation and some astrophysical terms used throughout the text. ... while section \ref{conclusions} sums up what was learned from the project and looks to the future for what should be studied next.