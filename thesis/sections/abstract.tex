Hydrodynamical cosmological simulations are powerfull tools in the study of galaxy formation and evolution, and the newest suite of state-of-the-art simulations like IllustrisTNG are pushing the boundaries of modern astrophysics. At the same time, large observational surveys of galaxies in the nearby universe have increased our understanding of what properties and distributions is expected in a large galaxy population. Comparing observations and simulations is not always straightforward, and the literature contains a multitude of methods and varying results. In this work, the methods used to calculate stellar and halo mass, characteristic size, velocities and color from the IllustrisTNG simulation are studied and compared against each other. It is found that stellar mass and effective radius is sensitively dependent upon the definition of a galaxy's size, while velocity and color estimates are not effected by any of the studied imposed galaxy size restrictions. The scaling relations at $z = 0$ related to these properties are also compared against the SAMI observational data.  ... How specific should I be in the abstract?