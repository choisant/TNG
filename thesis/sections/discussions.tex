%summary

%what was done
In this work the IllustrisTNG TNG-100 simulation has been analysed to extract information about the statistics of galaxy properties. The properties studied are stellar mass, halo mass, half-mass radius, velocities and color. Several different methods of calculating these properties has been employed, and are compared against each other as well as against the SUBFIND group catalog. Finally the TNG results were compared against the SAMI observational data as well as some auxillary observational results to evaluate the efficiency of TNG in reproducing galaxy properties.

\subsection{Discussion and summary}
It was found that TNG stellar mass estimates are highly dependent upon the way that the galaxy size is defined. For the largest galaxies ($M_{halo} > 10^{13.5} M_\odot$), stellar mass varies by as much as 40 \% depending on the aperture within which the mass is calculated. The reason for this is that we are ``cutting off'' the continous stellar particle distribution in the halo at the aperture radius. For large diffuse galaxies that can extend over 100 kpc from the center of the halo, the effect will be most pronounced, and this is what we are indeed finding. A direct comparison of stellar masses found in galaxies between TNG and other studies would therefore be significantly affected by the choice of how stellar mass is defined, and this must be taken into account when doing any kind of comparison. 


The stellar-to-halo mass relation for TNG and three recent observational results, \textcite{Behroozi2019} and \textcite{Zanisi2019} were then compared. TNG shows excellent agreement for the SHMR compared to observational results in the mass range $10^{9.5} M_\odot < M_\ast < 10^{11} M_\odot$. However, in the high mass range $M_{halo} > 10^{13} M_\odot$ TNG SHMR has a steeper slope than that found in observations. \textcite{Kravtsov2018} suggest that works like \textcite{Behroozi2019} and \textcite{Zanisi2019} underestimate the stellar mass, and their results for $M_{halo} > 10^{13.5} M_\odot$ halos are more similar to TNG $M^{SUBF}_\ast$. The slope of TNG is still too steep however. These results reproduce the findings in \textcite{Pillepich2017}, where also the galaxy stellar mass function is studied and compared to observational results. They show that there is an overabundance of massive galaxies in the simulations, and that galaxy stellar mass definitions have a large impact on the size of this mismatch. They also point out the intrinsic difficulty in separating the main galaxy's stellar mass from the inter-cluster light of the largest halos, and advocate that this should not be attempted. Stopping star formation in simulations is very difficult, and it requires a fine balance between allowing smaller galaxies to grow while quenching the star formation in the largest galaxies. There is therefore a good chance that TNG massive galaxies have grown too large \parencite[as suggested by][]{Vazquez2020}, but there is also a chance that observations are simply missing the low surface brightness outer parts of massive elliptical galaxies \parencite[see e.g.,][]{DSouza2015}. A suggestion to mitigate this issue is to not try to actually measure the entire galaxy, but use a stellar mass measurement within a fixed aperture in physical kiloparsecs. That would be a clear definition, and should be possible to do for both observational data and simulations. This would be an interesting project for future research.


The stellar half-mass radii were then estimated using the different stellar mass definitions (based on different galaxy size assumptions). The results as shown in Figure \ref{SM_R_TNG} show a divergence in half-mass radius around $M^{SUBF}_\ast = 10{11} M_\odot$. This is a consequence of the huge difference in the different stellar mass definitions. Both 3D and projected radius were calculated, and it was found that scaling the 3D radius by a factor of $3/4$ as proposed by \textcite{Wolf2010} is an excellent approximation to the average calculated projected radius. 
The size-mass relation of TNG is almost flat at masses below $M^{10.5}_{\odot}$. This tells us that TNG galaxies become more dense as they grow in size, until they reach the characteristic stellar mass $M_\ast = M^{10.5}_{\odot}$, where they start to expand with increasing mass. This trend is much less profound in the observational SAMI data, which has a positive slope across the mass spectrum. It is important to note that there is a large scatter in both observation and simulation results as well as uncertainities which are not accounted for in this study. TNG galaxies in this flat regime have larger median effective radius than SAMI galaxies by about 0.1 dex (26 \%), but have a 25-75 \% spread of $\pm$ 0.1 dex. Keeping this in mind, the similarity in the stellar mass - effective radius relation between SAMI and TNG is remarkable. 

There is however a larger difference when looking at early and late type galaxies seperately. It was also found that this relation is sensitively dependent on the selection criteria for morphology classification, both for TNG and SAMI. The largest difference for TNG data is in late type galaxies, where the criteria that should produce the most disky galaxies give the most diverging results compared to the most disky SAMI sample by being $\sim 0.2$ dex (58 \%) smaller for $M_\ast \sim 10^{10.5} M_\odot$ compared to SAMI.

A very important point to remember when considering these results are that they depend on the comparison between half-mass radius and half-light radius, which would be the same if the mass-to-light ratio was constant throughout the stellar mass profile of each galaxy. This does not appear to be the case, as it has been found that luminosity-weighted characteristic sizes are larger for late type galaxies \parencite{Sande2018}. They attribute this to the bulge part of a galaxy, which is redder and thus weighted less in the r-band than the bluer disk in the outer parts of the galaxies. A better comparison would then be to calculate half-light radius for TNG galaxies. This was done in \textcite{Genel2017}, where they found that $R_{hl}$ are similar to $R_{hm}$ up to the characteristic stellar mass of $M_\ast = 10^{10.5} M_\odot$, where they converge with the 3D half-mass radius $r_{hm}$ that are significantly larger than $R_{hm}$ across all masses (see Figure \ref{SM_R_TNG}). The apparent excellent match between TNG and observations in this work is therefore ...


Rotational velocity was also studied, and has no dependency on radius beyond $2 \times r_{hm}$, as expected. The Tully-Fisher relation of TNG late-type galaxies gives similar values as observations while having a slightly shallower slope. As the late type galaxies in the TNG sample generally do not exceed $M_\ast = 10^{11} M_{\odot}$, we do not see any difference in the TFR for stellar mass definitions $M_\ast^{15r200}$ or $M_\ast^{SUBF}$ compared to $M_\ast^{30kpc}$. Using the stellar mass measurement within $2 \times r_{hm}^{SUBF}$ would decrease the slope further, making for an even worse fit to the observational data. The SAMI fit is based on a sample of galaxies that span a larger stellar mass range than the TNG galaxies studied here, from $10^{7.5} M_{\odot} - 10^{11.5} M_{\odot}$. The TFR extends across the entire mass range though, with a higher scatter at low stellar masses, so it is still comparable to the TNG data which only spans a range of about $10^{9.5} M_{\odot} - 10^{11} M_{\odot}$. 


Next, the different particle's contribution to the total velocity dispersion in the subhalos was looked at, showing that gas particles have significantly lower velocities than stars and dark matter. The dark matter velocity dispersion is similar for the entire mass range of subhalos, while the stellar velocity dispersion was lower for both the smallest and the largest galaxies, with a maximum for galaxies with stellar mass $10^{10.5} M_\odot$. The stellar velocity dispersion has little dependency on galaxy aperture size down to $r^{SUBF}_{hm}$. TNG galaxies have a similar slope in the Faber-Jackson relation compared to observations, but has a lower zero point by about 0.1 dex.  From this, it would seem that velocity dispersions in TNG are lower than those seen in observations at redshift $z=0$. Based on the above analysis, it does not seem like this can be attributed to projection effects or the size of the volume within the velocity dispersion is calculated, but rather the velocities of the simulated stellar particles are in general lower than that which is observed in the stars of real elliptical galaxies. This is in agreement with the results of \textcite{Sande2018}, as they find simulated velocity dispersions to be lower than in observations. 


Finally, TNG produces a distinctly bimodal galaxy color distribution in the g-i color, as already determined by \textcite{Nelson2017}. The color distribution of the subhalos was not affected by any of the studied aperture sizes. Bimodality matches well with galaxy morphology classification, even with the least strict criteria where only sSFR was considered. Compared to SAMI observations the color distribution may be too binary as the gradual increase in redness with increasing stellar mass which is found in the observations is missing in the simulation data.  This is also mentioned by \textcite{Nelson2017}, who interestingly suggests that this is due to their choice of a 30 kpc aperture which excludes parts of the larger galaxies. This turns out to not be the case as mentioned earlier. Their other suggested solution was that there is a discrepancy in the relatively simple TNG dust modeling, which then seems to be plausible. TNG galaxies are also slightly bluer than the SAMI galaxies, reflecting the missing reddening with stellar mass within the two galaxy populations.


To summarise, the main findings are as follows.
%Conclusions bullet points
\begin{itemize}
	\item Of the properties studied here, the SUBFIND values are great to use for rotational velocity and color. The SUBFIND velocity dispersion measurement should not be used as a proxy for gas velocity dispersion, and is not a very precise estimate for stellar velocity dispersion either. SUBFIND values should be used with caution for stellar mass and half-mass radius for halos with a total mass larger than $10^{13} M_\odot$.
	\item Comparing TNG to observations, the TNG SHMR is very similar to observations for halo masses up to $10^{13} M_\odot$ where it becomes steeper. The size-mass relation shows excellent agreement for the entire galaxy population, but shows differences when separated into early and late type galaxies. The TFR of TNG has a shallower slope than observations, but values fall within the uncertainties. The FJ relation of TNG and observations has a remarkably similar slope, bu TNG has a slightly lower zero-point. The color bimodality in TNG is in good agreement with observations, although the slope in the relation seems too flat. All in all, the scaling relations of TNG exhibit the expected trends found in observational data, despite some discrepancies which do not seem to be related to the way in which the properties are calculated.
	\item Galaxy morphology classification should be treated carefully...

\end{itemize}

\subsection{Reflection and way forward}
The newest set of hydrodynamical cosmological simulations like TNG are so good at recreating the structures and properties of the Universe at cosmological scales that comparisons against observations are becoming more and more relevant and useful. This also means that observational and numerical astrophysicists must become even better at doing these kinds of comparisons in a fair and meaningful way. Efforts are being made in this direction, and especially \textcite{Sande2018} stands out as going to great lengths in reproducing the observational methods of calculating size and kinematic properties for the three different simulations they study. They do not however mention much about stellar mass measurements, and so a study similar to this on several other properties like stellar mass and color would be a great contribution towards this goal.

%reflection and way forward
Explicitly stating all definitions and methods used in a scientific work is extremely important for it to be relevant in the greater scope of astrophysics research. As cosmological simulations essentialy are ``black boxes'' to outside users, the way that the data is post-processed and interpreted is non-trivial in every sense, especially when comparing against observational data. Standards are quite different in the observational and numerical sections of astrophysics, so it can be hard to make meaningful comparisions (especially as someone new to the field). Developing a standard method of calculating properties and comparing them would be a step in the direction of making it easier to navigate the increasingly large amount of research done in the field of galaxy formation and evolution. In this age of digital revolution, huge amounts of data are acquired each year, both from the development of better observational instruments, creating new ways of analysing old data and from numerical experiments like IllustrisTNG. This means that new research is constantly being published, and it is hard to keep up with the inflow of information. That only makes it even more important to make sure that our works are easily reproduceable and that all methods and definitions are unambigously defined, preferably accompanied by a reflection on the impact of those choices. This will lead to more clarity, and eventually to a much more comparable set of works published in the future.