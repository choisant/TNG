\subsection{Stellar mass and half-mass radius}

Using the cut in galaxy size described in section \ref{galaxy_size} with $R_{gal} = 0.15 \times R_{200}$ the stellar mass in galaxies are found to be similar to those of SUBFIND up to about $M_* = 10^{10.5} M_{\odot}$. For more massive galaxies, the stellar mass becomes rapidly smaller compared to those of SUBFIND, reaching a fractional difference of almost 15\% at $M_* > 10^{11} M_{\odot}$. This of course translates into a difference in half-mass radii as well, although this trend is even steeper. For stellar masses larger than $10^{11} M_{\odot}$ the stellar half-mass radius is over 20\% smaller. Plotting the stellar mass as a function of the half-mass radius for both SUBFIND and the particles, it is clear that in the high-mass end of the spectrum the latter has a higher slope. As these galaxies are expected to be mostly early types, we might then experience a change in the slope of the fundamental plane.


\subsection{Kinematics}

Looking at the velocity dispersion for early type galaxies, there is a significant scatter about the one-to-one line. For the stellar mass range of $10^{10} M_{\odot} < M_* < 10^{11} M_{\odot}$ , the velocity dispersions calculated using the particles are both smaller and larger than those found by \texttt{SUBFIND}. At larger masses, the velocity dispersion is smaller. Now there are several variables which might affect this result. Firstly the galaxy size compared to using the entire subhalo, and secondly the method of simulating the projected velocity dispersion. We have therefore calculated the velocity dispersion for the entire subhalo using the projection method and also calculated it within the galactic radius without projection.

The rotational velocities at a radius of $2.2 \times R_{e}$ for late-type galaxies is behaving entirely as expected. It is similar to that of \texttt{SUBFIND}, just slightly smaller. 

\subsection{Color bimodality}

Completely similar.